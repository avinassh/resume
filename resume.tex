%% start of file `resume.tex'.
%% Copyright 2016 Avinash (hi@avi.im).

\documentclass[11pt, a4paper, sans]{moderncv} % possible options include font size ('10pt', '11pt' and '12pt'), paper size ('a4paper', 'letterpaper', 'a5paper', 'legalpaper', 'executivepaper' and 'landscape') and font family ('sans' and 'roman')

% moderncv themes
\moderncvstyle{classic} % style options are 'casual' (default), 'classic', 'oldstyle' and 'banking'
\moderncvcolor{grey} % color options 'blue' (default), 'orange', 'green', 'red', 'purple', 'grey' and 'black'
%\renewcommand{\familydefault}{\sfdefault} % to set the default font; use '\sfdefault' for the default sans serif font, '\rmdefault' for the default roman one, or any tex font name
\nopagenumbers{} % uncomment to suppress automatic page numbering for CVs longer than one page

% character encoding
%\usepackage[utf8]{inputenc} % if you are not using xelatex ou lualatex, replace by the encoding you are using

% adjust the page margins
\usepackage[scale=0.8]{geometry}
\usepackage{fontawesome}
%\setlength{\hintscolumnwidth}{3cm} % if you want to change the width of the column with the dates
%\setlength{\makecvtitlenamewidth}{10cm} % for the 'classic' style, if you want to force the width allocated to your name and avoid line breaks. be careful though, the length is normally calculated to avoid any overlap with your personal info; use this at your own typographical risks...

% personal data
\firstname{Avinash}
\familyname{Sajjanshetty}
\address{Bangalore, India}
\mobile{+91-944-838-3875}
\email{hi@avi.im}
\extrainfo{\faGithub \href{https://github.com/avinassh}{avinassh}} 


\begin{document}

\makecvtitle

\section{Experience}
\cventry{Jun 2013 -- March 2014}{Project engineer}{VLEAD}{Hyderabad}{}{Virtual Labs Engineering and Architecture Division (VLEAD), based in IIIT-Hyderabad, is one of the several teams working on the Govt. of India funded 'Virtual Labs' project. The project’s goal is to build labs in engineering and sciences providing real lab experience to all students with as minimum requirements as a computer system and an internet connection.\newline{}
\textbf{Major works}:%
\begin{itemize}%
\item \href{https://github.com/vlead/ovpl}{OVPL} - One VM Per Lab is a system which enables developers one click deployment of their Virtual Labs. I was one of the four contributors, worked on writing servers using Tornado and I improved system performance by 40\%. The project is opensource and can be found \href{https://github.com/vlead/ovpl}{here}.
\item {UIR} - UIR is a responsive UI framework for Virtual Labs. It is written mostly using CSS3 and bit of LESS. I conceived of, designed and created the UI and framework, which led to 67\% increase in site traffic and 92\% increase in performance.
\item \href{https://github.com/vlead/simo}{SIMO} - SIMO (Seamless Integration of MirrOring and syncing) is bot which enables syncing of Bazaar, Git and SVN repositories to a remote Git server, without actually converting repositories of different version control to Git. The project is opensource and can be found \href{https://github.com/vlead/simo}{here}.
\end{itemize}}


\section{Technical Skills}
\cvcomputer
{\textbf{Programming languages}}{Python, C, C++, \LaTeX}
{\textbf{Operating systems}}{Linux, OS X, Windows}
\cvcomputer
{\textbf{DBMS}}{MongoDB, MySQL}
{\textbf{Web}}{HTML, CSS, Javascript}
\cvcomputer
{\textbf{Frameworks}}{Tornado, Flask, Bootstrap, Pure}
{\textbf{Tools}}{GIMP, TeXWorks, Emacs, Version Control (Git, Bazaar, SVN)}
{}

\section{Education}
\cventry{}{Bachelors of Engineering(CS\&E)}{}{VTU University}{June 2012}{}


\section{Extras}
\cvlistitem{Actively participating in an open source project which helps people maintain their products wishlist, supporting multiple Indian e-commerce sites. }
\clearpage

\end{document}
%% end of file `resume.tex'.
